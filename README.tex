% Autogenerated translation of README.md by Texpad
% To stop this file being overwritten during the typeset process, please move or remove this header

\documentclass[a4paper, 12pt]{book}
\usepackage{graphicx}
\usepackage[utf8]{inputenc}
\usepackage[a4paper]{geometry}
\usepackage{hyperref}
\pagestyle{plain}
\begin{document}

\chapter*{Ball and Beam Control Demonstration}

\emph{Ball and Beam Control Demonstration} is an interactive demonstration of feedback control for a ball and beam system. It was written in Matlab to demonstrate proportional and proportional-derivative control in the classroom and for homework exercises. It was originally written in 2001, and revised in 2011 for the then-current version of Matlab. The github repository was created in 2015.

\texttt{$<$img align="center" src="https://raw.githubusercontent.com/jckantor/ESTM60203/master/notebooks/img/LPprob01.png" alt="" title=""/ width="480"  $>$}

\section*{Operation}

\subsection*{Installation}

The demonstration is contained in a single Matlab file \texttt{ballbeam.m}. 

The \texttt{ballbeam.m} file is a function that can be run with no arguments. Start the demo by entering \texttt{ballbeam} at the Matlab command prompt.

\subsection*{User Interaction}

The demo starts up in manual mode. The vertical slider moves the end of the beam up and down. The bottom slider adjusts the setpoint indicated by the red marker. Push the \texttt{Run} button to start the simulation. The ball will roll back and forth on the beam as you move the beam position up and down. The simulation is stopped by pushing the |Stop| button or when the ball rolls off either end of the beam. A summary of the simulation results is shown in the plot.

\begin{enumerate}
\item Exercise: Move the setpoint to a new place, press \texttt{Run}, and manipulate the beam position to settle the ball over the new setpoint.
\end{enumerate}

\section*{Dynamics}

 (http://nbviewer.ipython.org/github/jckantor/Ball-and-Beam-Control-Demonstration/blob/master/Dynamics.ipynb)

\end{document}
